
%------------------------------------------------------------------------- 
\Section{Project Approaches}
\label{sec:tolerance} 	
We have presented trace-based JIT compilation as an adaptive optimization method for LLVM which is the technique used by virtual machines to optimize the execution of a program at runtime. Our implementation uses a hybrid mode of execution approach: the VMKit starts running in a fast-starting byte-code interpreter. As the program runs, the VMKit identifies hot (frequently executed) byte-code sequences, records them, and compiles them to fast native code. This method will be called as sequence of instruction, a trace. We have made this design decision which is based on the assumptions that program are spend most of their time in hot loops. Further, trace-based JIT can greatly speed up programs if they spend most of their time in loops where they take similar code paths.