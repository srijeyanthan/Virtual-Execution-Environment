%------------------------------------------------------------------------- 
\Section{Internal Mechanisms Study -JIT compilation with LLVM}
\label{sec:concurrency}
The LLVM Project is a collection of modular and reusable compiler and tool chain technologies. All the libraries including VMKit have been implemented in C++. As VMKit is a substrate for VMs, the components integrated in order to develop VMKit should have a general purpose interfaces and functionalities. Therefore the compiler should have support for a general purpose instruction set to allow VMs to implement intermediate representations of arbitrary functions. LLVM JIT is exactly a suitable compiler for this task because of its inherent behavior without  imposing object model or type system.\\
Once the VM generates the intermediate representation with the help of interfaces provided by LLVM, that VM delegates the compilation to LLVM JIT to generate the native code. LLVM JIT optimizes all methods to the same degree without considering the frequency of the number of time each method get called. This behavior is called the compilation without adaptive optimization. As a result, the initialization time get higher and the runtime also shows some slowness. 
Our proposed method would be  an extension for LLVM Execution engine. once JIT compiler is triggered, it reads the byte code and creates Intermediate Representation code (LLVM IR). All the generated LLVM IR will then be optimized by the LLVM optimizer by removing dead code, constant propagation etc. Finally LLVM IR will produce machine code optimized for a specific CPU.

